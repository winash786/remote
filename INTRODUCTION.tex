\chapter{INTRODUCTION}
\section{BASIC INTRODUCTION}
Tungsten has high mechanical and physical properties, and has the highest melting point of all the non-alloyed metals and the second highest of all the elements after carbon. Tungsten is often brittle and hard to work in its raw state. It is generally used in electrical applications, but its compounds and alloys are used in numerous applications, most notably in light bulb filaments, X-ray tubes (as both the filament and target) and super alloys. It also used in aerospace, aeronautic, biomedical applications etc.

Ti-6Al4V or simply Ti64 is an alloy of Ti with 6\% of Aluminum and 6\% of Vanadium which is difficult to machine. Carbide tools, ceramic tools, CBN, PCD etc. can be used to machine the Ti alloy. This study concentrate on machining of Ti64 using PCD tool. Natural diamond has the 10000HV which is hardest but not commonly used because of its cost. PCD is synthetic diamond with hardness 8000HV. PCD tools are used to machine Ti, Ni, Fe and their alloys.

It is also used in many practical cases of super finishing, such as, in Al, Mg and their alloys, wood etc. In these applications the chip-tool temperature rarely is higher than 300$^\circ$ C, owing to the lower melting points of these materials. When it comes to applications for others metals with a melting point above 1000 $^\circ$ C, there arises the need for special techniques for the lubrication and cooling of the interface chip-tool. The diamond  changes to graphite  at temperatures around 750$^\circ$ C for a period that lasts longer than 1.5 min, and under such conditions tool wear occurs quickly.

